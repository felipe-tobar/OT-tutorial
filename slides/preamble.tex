\usetheme{%
%Berlin%
%Antibes%
%Boadilla%
%Copenhagen%
%Darmstadt%
%Frankfurt%
%Ilmenau%
%JuanLesPins%
%Madrid%
%Warsaw%
%Marburg%
%Boadilla%
simple%
}


%\usepackage{pgfpages}
%\pgfpagesuselayout{4 on 1}[a4paper,border shrink=5mm]
%\usepackage{scalefnt}
%\usepackage[version=4]{mhchem}
%\newcommand{\bftab}{\fontseries{b}\selectfont}
\usepackage{multirow}
\newcommand{\rot}[1]{\multirow{9}{*}{\rotatebox[origin=c]{90}{#1}}}
%\mode<presentation>


%\usecolortheme[RGB={15,90,0}]{structure}
%\usecolortheme[RGB={20,100,0}]{structure}
%\usecolortheme[RGB={10,80,0}]{structure}

\useinnertheme{circles}
%\useoutertheme{miniframes}
%\useoutertheme{shadow}
%\useoutertheme[subsection=false]{smoothbars}
%\useoutertheme{infolines}

% \usepackage[brazil]{babel}
%\usepackage[T1]{fontenc}
%\usepackage{ae} 
\usepackage[utf8]{inputenc} 
\usepackage[T1]{fontenc}


\usepackage{media9}      

\usepackage[onecol]{hackthefootline}
%\htfconfig{title=short, framenrs=counter, atsep=space}
\usepackage{algorithm}
\usepackage{algorithmic}
\usepackage{graphics}
\usepackage{graphicx}
%\usepackage[caption=false]{subfig}
\usepackage{caption}
%\usepackage{subcaption}
\usepackage{array} %for vertical thick lines in tables
\usepackage{multirow} %multirow tables
\usepackage{nicefrac} %for fractions like 1/4
\usepackage{booktabs}
\usepackage{multicol}
\usepackage{amssymb}
\usepackage{amsmath} 
\usepackage[makeroom]{cancel}
\usepackage{bm}
\usepackage[group-separator={,},separate-uncertainty]{siunitx}
\usepackage{xcolor}
\usepackage[export]{adjustbox}

%\usefonttheme[onlymath]{serif}
%\usepackage{lmodern}
\usepackage{makecell}
%% Commande TikZ
\usepackage{tikz}
\usepackage{tikz,pgf,tikz-3dplot}
\usepackage{pgfplots}
\pgfplotsset{compat=1.16}
\usepackage[beamer]{hf-tikz}
\usetikzlibrary{bayesnet,shapes,arrows,positioning}
\tikzstyle{arrow} = [thick,->,>=stealth]
\tikzset{
	%Define standard arrow tip
	>=stealth',
	%Define style for boxes
	punkt/.style={
		rectangle,
		rounded corners,
		draw=black, very thick,
		text width=6.5em,
		minimum height=2em,
		text centered},
	% Define arrow style
	pil/.style={
		->,
		thick,
		shorten <=2pt,
		shorten >=2pt,}
}


\tikzstyle{rempliRed}=[red!30, fill=red!30]
\tikzstyle{ecritRed}=[red!60, fill=red!60]
\tikzstyle{rempliBlue}=[blue!30, fill=blue!30]
\tikzstyle{ecritBlue}=[blue!60, fill=blue!60]

\definecolor{color1}{rgb}{0.25, 0, 0.75}
\definecolor{color2}{rgb}{0.5, 0, 0.5}
\definecolor{color3}{rgb}{0.75, 0, 0.25}

\pgfplotsset{
    dirac/.style={
        mark=triangle*,
        mark options={scale=2},
        ycomb,
        scatter,
        visualization depends on={y/abs(y)-1 \as \sign},
        scatter/@pre marker code/.code={\scope[rotate=90*\sign,yshift=-2pt]}
    }
}

\def\mathunderline#1#2{\color{#1}\underline{{\color{black}#2}}\color{black}}

\DeclareMathOperator{\Tr}{\mathrm{Tr}}
\DeclareMathOperator{\KL}{\mathrm{KL}}
\usepackage{dsfont}


%Felipe's packages


% Definitions
\newcommand*{\defeq}{\stackrel{\text{def}}{=}}
\newcommand{\tr}[1]{\operatorname{tr}\left(#1\right)}

\newcommand{\MOGP}{\ensuremath{\mathcal{MOGP}}}
\renewcommand{\L}{{\mathbb L}}
\def\data{\text{data}}
\def\sn{\sigma_{\text{noise}}}
\def\eye{\mathbf{I}}
\def\cN{{\mathcal{N}}}
\def\cX{{\mathcal{X}}}
\def\cY{{\mathcal{Y}}}
\def\dxi{{\text d} \xi}
\def\N{{\mathbb{N}}}
\def\m{{\mathbf{m}}}
\def\NLL{{\text{NLL}}}
\newcommand{\eps}{\varepsilon}
\def\R{{\mathbb{R}}}
\def\MVN{{\text{MVN}}}
\def\GP{{\mathcal {GP}}}
\def\y{{\mathbf{y}}}
\def\t{{\mathbf{t}}}
\def\x{{\mathbf{x}}}
\def\xizi{ \xi_0^{(i)} }
\def\u{{\mathbf{u}}}
\def\z{{\mathbf{z}}}
\def\X{{\mathbf{X}}}
\def\K{{\mathbf{K}}}
\def\dtau{{\text d\tau}}
\def\td{{\text d}}
\newcommand{\id}{\text{id}}


\newcommand{\fourier}[1]{\mathcal{F} \left\{#1\right\}}
\newcommand{\WF}[2]{\operatorname{WF}\left([#1],[#2]\right)}

\newcommand{\afourier}[1]{\mathcal{F}^{-1} \left\{#1\right\}}
\newcommand{\srp}[4]{\operatorname{simrect}_{#1,#2,#3}\left(#4\right)}
\newcommand{\rect}[1]{\text{rect}\left(#1\right)}
\def\sinc{ \operatorname{sinc} }
\newcommand{\SK}[1]{\operatorname{SK}\left(#1\right)}
\newcommand{\V}[2]{\mathbb{V}\left[#1,#2\right]}
\newcommand\eqdef{\overset{\mathrm{def}}{=}}
\newcommand{\GSK}[1]{\operatorname{GSK}\left(#1\right)}
\newcommand{\GSKN}[1]{\operatorname{GSK}_N\left(#1\right)}
\DeclareMathOperator\supp{supp}
\DeclareMathOperator{\argmin}{argmin}
%colours
\definecolor{navy}{HTML}{000080}
\providecommand{\rbullet}{{\textcolor{red}{{\bf \bullet}}}}
\providecommand{\bbullet}{{\textcolor{navy}{{\bf \bullet}}}}
\definecolor{clay}{RGB}{190,22,34}%
\providecommand{\bred}[1]{\textcolor{clay}{{\bf#1}}}
\providecommand{\bgreen}[1]{\textcolor{green}{{\bf#1}}}
\providecommand{\bteal}[1]{\textcolor{teal}{{\bf#1}}}
\providecommand{\bblue}[1]{\textcolor{navy}{{\bf#1}}}
\providecommand{\red}[1]{\textcolor{clay}{#1}}
\providecommand{\blue}[1]{\textcolor{navy}{#1}}
\providecommand{\bbullet}{{\textcolor{navy}{{\bf \bullet}}}}




%% Commande TikZ
\tikzstyle{rempliRed}=[red!30, fill=red!30]
\tikzstyle{ecritRed}=[red!60, fill=red!60]
\tikzstyle{rempliBlue}=[blue!30, fill=blue!30]
\tikzstyle{ecritBlue}=[blue!60, fill=blue!60]

\newcommand{\er}[1]{{\color{red!60} #1}}
\newcommand{\eb}[1]{{\color{blue!60} #1}}


% Commandes
\newcommand{\bigslant}[2]{{\raisebox{.2em}{$#1$}\left/\raisebox{-.2em}{$#2$}\right.}} 
\newcommand{\Z}{{\mathbb Z}}
\renewcommand{\P}{{\mathbb P}}
\newcommand{\G}{{\mathbb G}}
\renewcommand{\L}{{\mathbb L}}
\newcommand{\T}{{\mathbb T}}
\newcommand{\Q}{{\mathbb Q}}


\newcommand{\PP}{{\mathcal P}}
\newcommand{\HH}{{\mathcal H}}
\newcommand{\DD}{{\mathcal D}}
\newcommand{\BB}{{\mathcal B}}
\newcommand{\MM}{{\mathcal M}}
\newcommand{\XX}{{\mathcal X}}
\newcommand{\YY}{{\mathcal Y}}
\newcommand{\NN}{{\mathcal N}}
\newcommand{\LL}{{\mathcal L}}
\newcommand{\1}{{\mathds 1}}

\newcommand{\bX}{\boldsymbol{X}}
\newcommand{\bY}{\boldsymbol{Y}}
\newcommand{\bnu}{\boldsymbol{\nu}}
\newcommand{\bmu}{\boldsymbol{\mu}}
\newcommand{\boldeta}{\boldsymbol{\eta}}
\newcommand{\bsigma}{\boldsymbol{\sigma}}
\newcommand{\balpha}{\boldsymbol{\alpha}}
\newcommand{\bfun}{\boldsymbol{f}}
\newcommand{\bGamma}{\boldsymbol{\Gamma}}
\newcommand{\br}{\boldsymbol{r}}
\newcommand{\bq}{\boldsymbol{q}}
\newcommand{\bG}{\boldsymbol{G}}
\newcommand{\tH}{\tilde{H}}
\newcommand{\tg}{\tilde{g}}
\newcommand{\tS}{\tilde{\Sigma}}
\def\d{{\text d}}
\def\ba{ \boldsymbol{a} }
\def\bb{ \boldsymbol{b} }
\def\bx{ \boldsymbol{x} }
\def\by{ \boldsymbol{y} }




\newcommand{\CB}[1]{{\color{blue} #1}}
\newcommand{\CR}[1]{{\color{red} #1}}
\newcommand{\CV}[1]{{\color{green} #1}}
\newcommand{\CVI}[1]{{\color{violet} #1}}
\newcommand{\CP}[1]{{\color{purple} #1}}
\newcommand{\CO}[1]{{\color{orange} #1}}
\newcommand{\CM}[1]{{\color{magenta} #1}}
\newtheorem{proposition}{Proposition}  
\newtheorem{remark}{Remark}  



%imports
\usepackage[export]{adjustbox}
\usepackage[absolute,overlay]{textpos}
\setlength{\TPHorizModule}{\textwidth}
\setlength{\TPVertModule}{\textwidth}
\usepackage{soul}

% Pausing in the align environment
\makeatletter
\let\save@measuring@true\measuring@true
\def\measuring@true{%
  \save@measuring@true
  \def\beamer@sortzero##1{\beamer@ifnextcharospec{\beamer@sortzeroread{##1}}{}}%
  \def\beamer@sortzeroread##1<##2>{}%
  \def\beamer@finalnospec{}%
}
\makeatother

%listing package para código
\usepackage{listings}

\definecolor{codegreen}{rgb}{0,0.6,0}
\definecolor{codegray}{rgb}{0.5,0.5,0.5}
\definecolor{codepurple}{rgb}{0.58,0,0.82}
\definecolor{backcolour}{rgb}{0.95,0.95,0.92}
 
\lstdefinestyle{mystyle}{
	xleftmargin=0.05\textwidth,
	linewidth=0.95\textwidth,
    backgroundcolor=\color{backcolour},   
    commentstyle=\color{codegreen},
    keywordstyle=\color{magenta},
    numberstyle=\tiny\color{codegray},
    stringstyle=\color{codepurple},
    basicstyle=\ttfamily\scriptsize,
    breakatwhitespace=true,         
    breaklines=true,                 
    captionpos=b,                    
    keepspaces=true,                 
    numbers=left,                    
    numbersep=5pt,                  
    showspaces=false,                
    showstringspaces=false,
    showtabs=false,                  
    tabsize=2
}
 
\lstset{style=mystyle}